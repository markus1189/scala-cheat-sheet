\documentclass[10pt,landscape,a4paper]{article}
\usepackage[T1]{fontenc}
\usepackage[utf8]{inputenc}
\usepackage[ngerman]{babel}
\usepackage{tikz}
\usetikzlibrary{shapes,positioning,arrows,fit,calc,graphs,graphs.standard,trees}
\usepackage[nosf]{kpfonts}
\usepackage{multicol}
\usepackage{wrapfig}
\usepackage[top=2mm,bottom=13mm,left=2mm,right=2mm]{geometry}
\usepackage[framemethod=tikz]{mdframed}
\usepackage{microtype}
\usepackage{minted}
\usepackage{titlesec}
\usepackage{keystroke}
\usepackage{todonotes}
\newcommand{\TODO}[2][]{\todo[inline,color=green!40,linecolor=green!50,#1]{\small TODO #2}}

\renewcommand\familydefault{\sfdefault}

\definecolor{mygreen}{HTML}{4F8D4F}
\definecolor{myblue}{HTML}{3C6A6A}
\definecolor{myyellow}{HTML}{B18663}
\definecolor{myred}{HTML}{B16363}
\definecolor{sectionbg}{HTML}{D5D5D5}

\tikzstyle{every node}=[font=\footnotesize\sffamily]
\tikzstyle{sup}=[rectangle,draw,fill=myblue!50,rounded corners=.8ex]
\tikzstyle{sub}=[rectangle,draw,fill=mygreen!70,rounded corners=.8ex]
\tikzstyle{block} = [rectangle, draw, text width=2.5cm, text centered, rounded corners, minimum height=4em,fill=blue!20]
\tikzstyle{line} = [draw,thick, -latex']
\tikzstyle{cloud} = [draw, ellipse, text width=2.5cm, text centered]
\tikzstyle{edge from parent}=[<-,thick,draw]

\let\bar\overline

\def\firstcircle{(0,0) circle (1.5cm)}
\def\secondcircle{(0:2cm) circle (1.5cm)}

\colorlet{circle edge}{myblue}
\colorlet{circle area}{myblue!5}

\tikzset{filled/.style={fill=circle area, draw=circle edge, thick},
  outline/.style={draw=circle edge, thick}}

\pgfdeclarelayer{background}
\pgfsetlayers{background,main}

\everymath\expandafter{\the\everymath \color{myblue}}
\everydisplay\expandafter{\the\everydisplay \color{myblue}}

\renewcommand{\baselinestretch}{.8}
\pagestyle{empty}

\global\mdfdefinestyle{header}{%
  linecolor=gray,linewidth=1pt,%
  leftmargin=0mm,rightmargin=0mm,skipbelow=0mm,skipabove=0mm,
}

\newcommand{\header}{
  \vspace{1mm}
  \begin{center}
    \includegraphics[height=09mm]{codecentric.png}
  \end{center}
}

\makeatletter

\newcommand{\colorsection}[1]{\colorbox{sectionbg}{\parbox{\textwidth-215\fboxsep}{\centering#1}}}
\newcommand{\colorsubsection}[1]{\footnotesize\colorbox{sectionbg}{\parbox{\textwidth-250\fboxsep}{\centering#1}}}
\titleformat{\section}[block]{\color{myblue}\sffamily\small\bfseries\filcenter}{}{1em}{\colorsection}
\titleformat{\subsection}[block]{\color{myblue}\sffamily\small\bfseries\filcenter}{}{1em}{\colorsubsection}

\def\multi@column@out{%
  \ifnum\outputpenalty <-\@M
  \speci@ls \else
  \ifvoid\colbreak@box\else
  \mult@info\@ne{Re-adding forced
    break(s) for splitting}%
  \setbox\@cclv\vbox{%
    \unvbox\colbreak@box
    \penalty-\@Mv\unvbox\@cclv}%
  \fi
  \splittopskip\topskip
  \splitmaxdepth\maxdepth
  \dimen@\@colroom
  \divide\skip\footins\col@number
  \ifvoid\footins \else
  \leave@mult@footins
  \fi
  \let\ifshr@kingsaved\ifshr@king
  \ifvbox \@kludgeins
  \advance \dimen@ -\ht\@kludgeins
  \ifdim \wd\@kludgeins>\z@
  \shr@nkingtrue
  \fi
  \fi
  \process@cols\mult@gfirstbox{%
    %%%%% START CHANGE
    \ifnum\count@=\numexpr\mult@rightbox+2\relax
    \setbox\count@\vsplit\@cclv to \dimexpr \dimen@-1cm\relax
    \setbox\count@\vbox to \dimen@{\vbox to 1.5cm{\header}\unvbox\count@\vss}%
    \else
    \setbox\count@\vsplit\@cclv to \dimen@
    \fi
    %%%%% END CHANGE
    \set@keptmarks
    \setbox\count@
    \vbox to\dimen@
    {\unvbox\count@
      \remove@discardable@items
      \ifshr@nking\vfill\fi}%
  }%
  \setbox\mult@rightbox
  \vsplit\@cclv to\dimen@
  \set@keptmarks
  \setbox\mult@rightbox\vbox to\dimen@
  {\unvbox\mult@rightbox
    \remove@discardable@items
    \ifshr@nking\vfill\fi}%
  \let\ifshr@king\ifshr@kingsaved
  \ifvoid\@cclv \else
  \unvbox\@cclv
  \ifnum\outputpenalty=\@M
  \else
  \penalty\outputpenalty
  \fi
  \ifvoid\footins\else
  \PackageWarning{multicol}%
  {I moved some lines to
    the next page.\MessageBreak
    Footnotes on page
    \thepage\space might be wrong}%
  \fi
  \ifnum \c@tracingmulticols>\thr@@
  \hrule\allowbreak \fi
  \fi
  \ifx\@empty\kept@firstmark
  \let\firstmark\kept@topmark
  \let\botmark\kept@topmark
  \else
  \let\firstmark\kept@firstmark
  \let\botmark\kept@botmark
  \fi
  \let\topmark\kept@topmark
  \mult@info\tw@
  {Use kept top mark:\MessageBreak
    \meaning\kept@topmark
    \MessageBreak
    Use kept first mark:\MessageBreak
    \meaning\kept@firstmark
    \MessageBreak
    Use kept bot mark:\MessageBreak
    \meaning\kept@botmark
    \MessageBreak
    Produce first mark:\MessageBreak
    \meaning\firstmark
    \MessageBreak
    Produce bot mark:\MessageBreak
    \meaning\botmark
    \@gobbletwo}%
  \setbox\@cclv\vbox{\unvbox\partial@page
    \page@sofar}%
  \@makecol\@outputpage
  \global\let\kept@topmark\botmark
  \global\let\kept@firstmark\@empty
  \global\let\kept@botmark\@empty
  \mult@info\tw@
  {(Re)Init top mark:\MessageBreak
    \meaning\kept@topmark
    \@gobbletwo}%
  \global\@colroom\@colht
  \global \@mparbottom \z@
  \process@deferreds
  \@whilesw\if@fcolmade\fi{\@outputpage
    \global\@colroom\@colht
    \process@deferreds}%
  \mult@info\@ne
  {Colroom:\MessageBreak
    \the\@colht\space
    after float space removed
    = \the\@colroom \@gobble}%
  \set@mult@vsize \global
  \fi}

\makeatother
\setlength{\parindent}{0pt}

\mdfsetup{middlelinecolor=myblue, middlelinewidth=1pt, roundcorner=10pt}

\begin{document}
\setminted{fontsize=\footnotesize}

\small
\begin{multicols*}{4}

  \section{Variables}
\begin{minted}{scala}
var x = "mutable"
val y = "immutable"

// Initialized once on first access
lazy val z = "lazy"

// Pattern matching on left side
val (one,two) = ("one", '2')
\end{minted}

  \section{Methods}
\begin{minted}{scala}
def id(c: Char): Char = c

def add(n: Int, m: Int): Int = n + m

// Curried version (one argument per list)
def add(n: Int)(m: Int): Int = n + m

// By-name parameters, evaluates `a` twice
def twice[A](a: => A) = { a; a }

// Repeated Parameters (Varargs)
def many(cs: Char*): Seq[Char] = cs

// Calling a varargs method with a `Seq`
> many(Seq('1', '2') :_*)
\end{minted}

  \section{Classes}

\begin{minted}{scala}
// Implicit default constructor
// `AnyRef` plays the role of `Object`
class Foo extends AnyRef {
  val bar: Int = 42
  def foobar: Boolean = true
}

// Parameterized constructor
class Foo(msg: String) { ... }

// Additional constructor
class Foo(msg: String) {
  def this(n: Int) = this(n.toString)
}

// Can inherit from exactly one class
class Bar extends Foo("foo")
\end{minted}

  \section{Objects}

  \begin{mdframed}
    Objects can contain \textit{static} methods known from Java and
    play an important role in \texttt{implicit search} when used as a
    companion module.
  \end{mdframed}

\begin{minted}{scala}
object Foo {
  val hello: String = "Hello"
  def world: String = "World"
}

// Special case:
// class and object share name and source file
// => the object becomes a companion module
class Bar
object Bar { ... }
\end{minted}

    \section{Imports}

\begin{minted}{scala}
// Regular import
import scala.util.Try

// Import multiple
import scala.util.{Success, Failure}

// Import wildcard
import scala.util._

// Import under different name
import java.time.{Clock => JClock}

// Import everything except
import scala.util.{Try => _, _}

// Multiple import clauses in one line
import scala.util.Try, scala.util.Success
\end{minted}

  \section{For-loop and For-comprehension}

\begin{minted}{scala}
// for-loop
for (i <- 1 to 10) println(i)

// nested for-loop
for (i <- 1 to 10; j <- 1 to 10) println((i, j))

// for-comprehension
for (i <- 1 to 3; j <- 1 to i) yield i * y

// guards in for-loops and for-comprehensions
for (i <- 1 to 5 if i > 4) yield i

// curly braces for multiline expressions
for {
  i <- 1 to 5
  j <- 1 to i
  if j % 2 == 1
} yield i * j
\end{minted}
  \section{Pattern Matching}

\begin{minted}{scala}
arg match {
  // Variable Patterns
  case x =>

  // Typed Patterns
  case x: String =>

  // Literal Patterns
  case 42 =>

  // Stable Identifier Patterns
  case `foo` =>

  // Constructor Patterns
  case Foo(x,y) =>

  // Tuple Patterns
  case (x,y,z) =>

  // Extractor Patterns
  // (See 'Custom Extractors')
  case NumString(x) =>

  // Pattern Sequences
  case x1 +: x2 +: xs =>

  // Pattern Alternatives
  case true | 42 | "str" =>

  // Pattern Binders
  case pair@(x,y) =>

}
\end{minted}

  \section{Custom Extractors}

  \subsection{unapply with Boolean}

\begin{minted}{scala}
object Even {
  def unapply(n: Int): Boolean =
    n % 2 == 0
}
\end{minted}

\begin{minted}{text}
> 41 match { case Even() => '!' }
scala.MatchError: 41 ...
> 42 match { case Even() => "even!" }
res: String = "even!"
\end{minted}

  \subsection{unapply with Option}

\begin{minted}{scala}
object NumString {
  def unapply(s: String): Option[Int] =
    Try(s.toInt).toOption
}
\end{minted}

\begin{minted}{text}
> "42" match { case NumString(n) => n }
res: Int = 42
> "scala" match { case NumString(n) => n }
scala.MatchError: scala
\end{minted}

  \subsection{unapplySeq}

\begin{minted}{scala}
object Words {
  def unapplySeq(s: String): Option[Seq[String]] =
    Some(s.split("\\s+").to[Seq])
}
\end{minted}

\begin{minted}{text}
> "foo bar baz" match { case Words(ws) => ws }
res: Seq[String] = Vector("foo", "bar", "baz")
> "test" match { case Words("foo" +: _) => 1 }
scala.MatchError: test
\end{minted}

  \section{Type Parameters}
  \TODO{}

\begin{minted}{scala}
// Two type parameters A and B
def foo[A, B](x: Int) = ???

// Upper Bound, A has to be a subtype
def foo[A <: String]

// Lower Bound, A has to be a supertype
def foo[A :> String]

// View Bound (deprecated!)
def foo[A <% Ordered[A]](
  x: A, y: A): Boolean = x < y

// Context Bound
def foo[A:Ordering](x: A, y: A): Boolean = {
  import Ordering.Implicits._ // infix syntax
  x < y
}
\end{minted}

    \section{Implicits}
  There are two categories of places where Scala searches for implicits:
  \begin{enumerate}
  \item identifiers accessible without prefix at the call-site
  \item implicit scope: all companion modules of classes associated
    with the implicit's type
  \end{enumerate}

  \subsection{Uses of the implicit modifier}

\begin{minted}{scala}
// implicit values
implicit val n: Int = 42

// implicit conversions
implicit def f(n: Int): String = n.toString

// implicit classes
implicit class Wrapper[A](val a: A) {
  def printMe: Unit = println(a)
}

// implicit parameters
def foo(implicit ec: ExecutionContext) = ???
\end{minted}

%%%%%%%%%%%%%%%%%%%%%%%%%%%%%%%%%%%%%%%%%%%%%%%%%%%%%%%%%%%%%%%%%%%%%%%%%%%%%%%%
  \newpage{}
%%%%%%%%%%%%%%%%%%%%%%%%%%%%%%%%%%%%%%%%%%%%%%%%%%%%%%%%%%%%%%%%%%%%%%%%%%%%%%%%

  \section{Option, Either, Try}

  \subsection{Option}
  \begin{mdframed}
    Replaces \texttt{null}, there is only \textit{one} obvious
    reason for a missing value.
  \end{mdframed}
  \begin{center}
    \begin{tikzpicture}[
      level 1/.style={sibling distance=10em}]
      \node[sup]{Option[A]}
      child{node[sub] {None[Nothing]}}
      child{node[sub]{Some[A](value: A)}}
      ;
    \end{tikzpicture}
  \end{center}
\begin{minted}{scala}
val some: Option[Int] = Some(1)
val none: Option[Int] = None

// getOrElse
> some.getOrElse(42)
res: Int = 1
> none.getOrElse(42)
res: Int = 42

// fold
> some.fold("")(_.toString)
res: String = "1"
> none.fold("")(_.toString)
res: String = ""

// orElse
> some.orElse(none)
res: Option[Int] = Some(1)
> none.orElse(Some(42))
res: Option[Int] = Some(42)

\end{minted}
  \subsection{Either}
  \begin{mdframed}
    \textit{Domain} errors that have to be handled, there are
    \textit{multiple} reasons for an error.
  \end{mdframed}
  \begin{center}
    \begin{tikzpicture}[
      level 1/.style={sibling distance=10em}]
      \node[sup]{Either[A, B]}
      child{node[sub] {Left[A](value: A)}}
      child{node[sub]{Right[B](value: B)}}
      ;
    \end{tikzpicture}
  \end{center}
\begin{minted}{scala}
val right: Either[String, Int] = Right(1)
val left:  Either[String, Int] = Left("oops")

// getOrElse
> right.getOrElse(42)
res: Int = 1
> left.getOrElse(42)
res: Int = 42

// map on left and right side
> right.map(_ + 1)
res: Either[String, Int] = Right(2)
> right.left.map(_.length)
res: Either[Int, Int] = Right(1)
> left.left.map(_.length)
res: Either[Int, Int] = Left(4)
\end{minted}
  \subsection{Try}
  \begin{mdframed}
    Interact with Java / Legacy Code where exceptions are thrown, a
    means of last resort.
  \end{mdframed}
  \begin{center}
    \begin{tikzpicture}[
      level 1/.style={sibling distance=10em}]
      \node[sup]{Try[A]}
      child{node[sub] {Failure[A](e: Throwable)}}
      child{node[sub]{Success[A](value: A)}}
      ;
    \end{tikzpicture}
  \end{center}
\begin{minted}{scala}
> import scala.util.Try

> Try { "hello".toInt }
res: Try[Int] =
  Failure(java.lang.NumberFormatException)

> Try { "42".toInt }
res: Try[Int] = Success(42)
\end{minted}

  \section{Collections}
  \TODO{}

  \begin{itemize}
  \item hierarchy?
  \item mutable / immutable
  \item list vs vector
  \item map, set, ...
  \end{itemize}

  \section{Futures}

  \begin{mdframed}
    Don't blindly import Scala's default \texttt{ExecutionContext}, it
    is optimized for CPU-bound tasks.  Using it for blocking tasks
    will lead to starvation.
  \end{mdframed}

\begin{minted}{scala}
> import scala.concurrent._
> import ExecutionContext.Implicits.global
> import duration._

// Creating an asynchronous computation
> Future { 5 * 2 }
res1: Future[Int] = Future(<not completed>)

// Modify result with a pure function
> res1.map((n: Int) => n + 11)
res2: Future[Int] = Future(<not completed>)

// Use flatMap to chain Futures
> res2.flatMap((n: Int) => Future { n * 2 })
res3: Future[Int] = Future(<not completed>)

// Register callbacks
> res3.onComplete {
  case Success(r) => println(s"Success: $r")
  case Failure(e) => println(s"Failure: $e")
}

// Waiting for the result
// Avoid this, as much as possible
> Await.result(res3, 1.second)
res4: Int = 42
\end{minted}
  \subsection{Duration DSL}

\begin{minted}{scala}
> import scala.concurrent.duration._
> 5.seconds
res: FiniteDuration = 5 seconds
> 2.hours
res: FiniteDuration = 2 hours
> 100.millis
res: FiniteDuration = 100 milliseconds
\end{minted}

  \section{Types}

  \begin{center}
    \begin{tikzpicture}[auto,grow=right]
      \tikzstyle{level 1}=[sibling distance=16em,level distance=3em]
      \tikzstyle{level 2}=[sibling distance=4em,level distance=6em]
      \node [sup] (any) {Any}
      child{node [sup] (anyval) {AnyVal}
        child{node [sub] (nums) {Numbers}}
        child{node [sub] (char) {Char}}
        child{node [sub,fill=myred!60] (char) {Your AnyVal}}
        child{node [sub] (bool) {Boolean}}
      }
      child{node [sup] (anyref) {AnyRef}
        child{node [sub] (colls) {Collections}}
        child{node [sub,fill=myred!60] (clss) {Your Class}}
        child{node [sub] (str) {String}}
      };
      \node[sup,xshift=1cm,right of=clss](null){Null};
      \node[sup,xshift=4.5cm,right of=any](nothing){Nothing};
      \node[sup,above of=anyref](obj){Object};
      \path [line] (null) -- (str);
      \path [line] (null) -- (colls);
      \path [line] (null) -- (clss);
      \path [line] (nothing) -- (null);
      \path [line] (nothing) -- (bool);
      \path [line] (nothing) -- (char);
      \path [line] (nothing) -- (nums);
      \path [line,dashed] (obj) -- (anyref);
    \end{tikzpicture}
  \end{center}

  \section{IntelliJ Scala}

  \begin{tabular}{l l l}
    Reformat & \Ctrl{}\Alt{} & \keystroke{L} \\[1mm]
    Expand selection & \Ctrl{} & \keystroke{W} \\[1mm]
    Surround with & \Ctrl{}\Alt{} & \keystroke{T}
  \end{tabular}

  \subsection{Implicits}

  \begin{tabular}{l l l}
    Implicit parameters & \Ctrl{}\Shift & \keystroke{P} \\[1mm]
    Implicit conversions & \Ctrl{}\Shift & \keystroke{Q}
  \end{tabular}

  \subsection{Refactor}
  \begin{tabular}{l l l}
    Inline & \Ctrl{}\Alt{} & \keystroke{N} \\[1mm]
    Extract Variable & \Ctrl{}\Alt{} & \keystroke{V} \\[1mm]
    Extract Field & \Ctrl{}\Alt{} & \keystroke{F} \\[1mm]
    Extract Parameter & \Ctrl{}\Alt{} & \keystroke{P} \\[1mm]
    Extract Method & \Ctrl{}\Alt{} & \keystroke{M} \\[1mm]
    Copy & \Shift{} & \keystroke{F5} \\[1mm]
    Rename & \Shift{} & \keystroke{F6} \\
  \end{tabular}

  \subsection{Running \& Debugging}
  \begin{tabular}{l l l}
    Stop & \Ctrl{} & \keystroke{F2} \\[1mm]
    Debug & \Ctrl{} & \keystroke{F9} \\[1mm]
    Debug last & \Shift & \keystroke{F9} \\[1mm]
    Toggle breakpoint & \Ctrl{} & \keystroke{F8} \\[1mm]
    View breakpoints & \Ctrl{}\Shift & \keystroke{F8} \\[1mm]
    Temp. breakpoint & \Ctrl{}\Shift{}\Alt{} & \keystroke{F8}
  \end{tabular}
  \vspace{1cm}
\end{multicols*}
\end{document}
